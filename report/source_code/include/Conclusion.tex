% CREATED BY DAVID FRISK, 2016
\chapter{Conclusion}
The field of active particles, though being a new field, is widely researched across
multiple disciplines, such as statistical physics\cite{ramaswamy2010mechanics}, 
biology\cite{viswanathan2011physics}, robotics\cite{brambilla2013swarm}, social transport\cite{helbing2001traffic}, soft matter\cite{marchetti2013hydrodynamics} and biomedicine\cite{wang2012nano}. 

While most of the researches in the the field of active particles are often done in 
the microscale with microorganisms\cite{lauga2009hydrodynamics, cates2012diffusive, poon2013clarkia}.
There are yet grounds to be explored in the macroscale, which was the origin of this project.

What was seen here as the results of this project might strike to be somewhat intuitive 
and expected, with for example higher obstacle density or higher obstacle weight, 
leading to less motion for the active particles. Some results are less intuitive, 
such as higher number of active particles resulting in lower velocity for the active particles. 
However intuitive a result might be, it was interesting to go through the experiments to confirm 
the expectations.

Qualitative results was also presented as collective behaviors where we saw 
behaviors such as channel formation, reusing of channels, cooperative behaviors 
and other collective behaviors. These behaviors could only be observed clearly 
when the obstacle density was high.

Though it is obvious in hindsight that for example a bug will take a certain path 
or circling around one path or even pushing another bug because it has no where else 
to go, it is daunting that these behavior appears randomly when changing some parameters 
that do not have any obvious connection to these specific behaviors.        

Emergent behaviors is one of key properties of complex systems. 
It is powerful to think that a system with simple agents can exhibit complex behaviors when 
some parameter is being tuned. This is the heart of our nature with simple cells that follows 
simple rules, but together can accomplish something as complex as the human body\cite{hmelo2006understanding} or even 
the human brain\cite{meunier2009hierarchical}. Thought this field is still in its infancy, 
it is exciting to speculate what potential it has. The potential applications could be 
for examples drug delivery or elimination of cancer cells using nano robots. 
Fires could be put out by a swarm of drones, building could be built safer for evacuations using the knowledge of 
many particle systems simulation, or even designing autonomous robots for remote exploration such as rover in mars. 

With all being presented here, there is still works to be done and it is 
encouraged for future works to improve what was done in this project, continuing 
to find better and smarter ways to study active-passive particles systems in macroscale. 
And hopefully this project contributes to advancing the field of active matter.

%The behaviors of active particles exists in many scales, janus particles in nanoscales, 
%behaviour presents itself in different scales. In nanoscales with nanoparticles in [refer]
%that shows..... . In microscale with e-coli bacteria [refer] that behave similar too.... 
%And finally with this project on the macroscale using robots in an complex environment 
%to show that the system exhibit similar behaviour to both simulations and experiments 
%of different scales. 


