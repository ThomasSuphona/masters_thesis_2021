% CREATED BY DAVID FRISK, 2016
Collective behaviors of autonomous robots in complex environment\\
A study of emergent behaviors from many simple agents in a complex environment\\
THOMAS SUPHONA\\
Department of Applied Physics\\
Chalmers University of Technology \setlength{\parskip}{0.5cm}

\thispagestyle{plain}			% Supress header 
\setlength{\parskip}{0pt plus 1.0pt}
\section*{Abstract}
%Collective behaviors can be observe in nature with flocks of animals, birds swarming, schools of fish and even human crowds.
Collective behaviors or collective motion is a common phenomena in nature where multiple organisms in a system undergo 
ordered movements. This can be observed in different scales, from the microscale with bacteria swarming to 
the macro scale with for example flocks of birds, schools of fish and even human crowds and car traffic. 
%Shared between these phenomena is the property where each agent are capable of self-propulsion. These agents, 
%often referred as active particles, are able to take up energy from their surroundings and converting it 
%to directed motion.
All these systems are made up by self-propelling agents who are able to take up energy from their environment % Often modelled as active particles 
and converting it to directed motion.
%Because of this property of self-propulsion, their dynamics cannot be explained by mere thermal equilibrium. 
Because of this property of self-propulsion, their dynamics cannot be explained using conventional methods. 
%Although significant efforts have been made in trying to explain the behaviour of active-particles systems 
%from different perspective, ranging from, simulations, nano-particles, bacterias to robots, 
%the subject is not as widely studied from the macroscale. 
Although significant efforts have been made in trying to explain collective behaviors from different perspective, using 
simulation tools and study systems in different scales as mentioned before, the subject is not as widely studied from the 
macroscale, especially with artificially made systems.
%With a focus on the features of the interactions of 
%the active particles with a complex environment, this paper will reconstruct similar collective behaviours seen in nature, 
%using experiments and furthermore compare to simulations.
In this thesis, a macroscale system was design with the purpose of providing conditions for collective behaviors to 
emerge and study how the behaviors changes depending on the surrounding conditions. 
Battery powered robots were used as self-propelling agents and they were placed in a confined space filled 
with obstacles.
%It was shown that large amount of particles inside the system, both active and passive, 
%lead to on average less motion for the active particles.
It was shown that when the number of robot and obstacles inside the system is large, the robots movements were 
significantly restricted.
%The weight of the passive particles do also effect the average motions of the active particles where heavier passive particles 
%hinders the active particles significantly leading to them having lower average velocity.
The weight of the obstacles do also affect the average motions of the robots where heavier obstacles 
hinder the robots by creating blockage leading to the robots having lower average velocity.
%At certain configuration of the parameters, the active particles showed collective behaviours where they 
%for example form channels between the passive particles, making ''roads'' for other active particles to reuse, or 
%helping each other to move by pushing away chunks of passive particles or pushing onto each other.
At certain configuration of the parameters, the robots showed collective behaviors where they 
for example form channels between the obstacles, making ''roads'' for other robots to reuse, or 
helping each other to move by pushing away chunks of obstacles or pushing onto each other.
Even though these robots are simple agents, they have manage to manifest cooperative actions towards other agents. 
%The results presented here will hopefully lead to a better understanding in collective behaviours and 
%lay ground for future advancements in this field of nonequilibrium phenomena. 
%How these collective behaviours emerge is far from obvious, but any efforts made in advancing this 
%field of non-equilibrium phenomena are of great value, laying ground for future advancements and 
%applications.
% KEYWORDS (MAXIMUM 10 WORDS)
\vfill
Keywords: collective behaviors, complex environment, flocks, swarm.

\newpage				% Create empty back of side
\thispagestyle{empty}
\mbox{}
