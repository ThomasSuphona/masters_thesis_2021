% CREATED BY DAVID FRISK, 2016
\chapter{Results}
\section{Mean Square Displacement}

The main objective that was mentioned before is to try to observe how 
the behavior of the particles change with the parameters being used, 
the number of bugs, the number of passive particles and the weight of the passive particles. 
More specifically if any phase transition similar to \cite{nilsson2017metastable} can be observed.
A good way to start is to use the same quantitative measurement to see if there are any similarities.
Firstly, the mean square displacement (MSD) of the active particles is measured. Using MSD 
will give us an idea about the character of the active particles motion over time. 
The MSD is often used to check whether the motion of a system is due to diffusion or if there is 
a force contributing. The mean square displacement in 2-dimensions can be expressed as follows

\begin{align}
	MSD(\tau)   &=  \langle (x(t+\tau)-x(t))^2 + (y(t+\tau)-y(t))^2\rangle \label{eqn:2dim_msd}\\
		    &=  \dfrac{1}{N-\tau}\sum_{t=1}^{N}\left( x(\tau+t)-x(t)\right)^2 + \left( y(\tau+t)-y(t)\right)^2
\end{align}

What is to be expected is the result from \cite{nilsson2017metastable}
where the MSD for the active particles is higher at a low noise level, in our case 
low obstacle weight, and the MSD lower at a high noise level, high obstacle weight in our case. 
Other expected results from the experiment are also that MSD will be higher for cases with low obstacle density.

Firstly the active particles MSD is measured as a function of number of bugs.
In \cref{fig:msd_NB} the MSD of the active particles is calculated in 4 cases with 
different obstacle weights. 

\begin{figure}[htbp]
    \centering
    \includegraphics[width=.8\textwidth]{figure/MSD/NB_msd.pdf}
    \caption{MSD for the active particles in 4 different experiments where the number of 
    active particles is used as a parameter. As can be seen, the MSD is slightly lower when the number
    of active particles is high and this trend holds for different obstacle weights, 
    which is shown in $\textbf{(a)}$ with $m_{\text{passive}}=\SI{2e-3}{kg}$, 
    $\textbf{(b)}$ $m_{\text{passive}}=\SI{7e-3}{kg}$, 
    $\textbf{(c)}$ $m_{\text{passive}}=\SI{12e-3}{kg}$, 
    and $\textbf{(d)}$ $m_{\text{passive}}=\SI{17e-3}{kg}$.} 
    \label{fig:msd_NB}
\end{figure}

A general trend can be observed from \cref{fig:msd_NB} where the MSD is lower 
when the number of active particles is high, and inversely the MSD is higher 
when the number of active particles is low.

How the number of active particles affect the MSD is not obvious, compared to the 
obstacle weight as a parameter where there is a strong intuition of the fact that 
heavier obstacles will prevent the bugs from moving, hence lower MSD. 
There are though certain behaviors from the active particles that depend on 
their number and might in turn affect the MSD. This will be discussed further in 
\cref{collective_behaviors}\\

For the case of varying the weight of the obstacles, the result is shown in \cref{fig:msd_NW}.

\begin{figure}[htbp]
    \centering
    \includegraphics[width=.8\textwidth]{figure/MSD/NW_msd.pdf}
    \caption{MSD for the active particles as a function of obstacle weight. An 
    expected result where the MSD is lower when the obstacle weight is high. Since 
    it is harder for the active particles to push heavy obstacles, they will then 
    move less and this will lead to them having lower MSD. This same trend can be observed 
    with different number of active particles with $N_{\text{active}}=10$ in $\textbf{(a)}$, 
    $N_{\text{active}}=15$ in $\textbf{(b)}$, $N_{\text{active}}=20$ in $\textbf{(c)}$ and 
    $N_{\text{active}}=25$ in $\textbf{(d)}$.} 
    \label{fig:msd_NW}
\end{figure}

This shows that while keeping the number of obstacles and number of bugs fixed, the 
MSD is lower for the cases with high obstacle weight, this result was expected 
since it is harder for the bugs to push heavier obstacles which will resulting in them 
having lower speed throughout the experiment. The weight of the obstacles is a parameter 
that is suppose to mimic the orientational noise parameter in \cite{nilsson2017metastable} and to 
compare the result from this paper we can see in \cref{fig:simon_msd} that the 
curves is higher for the cases with low noise level $(\eta)$, similar with the 
experimental result in \cref{fig:msd_NW}. In the simulation from \citeauthor{nilsson2017metastable}, 
the active particles MSD transition from a superdiffusive motion at small times 
($\text{MSD}\propto\tau^{\alpha}$, $\alpha>1$ for small $\tau$), to diffusive motion at long times 
($\text{MSD}\propto\tau^{\alpha}$, $\alpha\leq1$ for large $\tau$). This similar behavior can also 
be observed in the experiment in \cref{fig:msd_NW} where, for small times the exponent $\alpha>1$ which 
suggest superdiffusion and $\alpha\leq1$ for large times suggesting diffusiv motions.


For the last case of studying MSD as a function of obstacles density, the result is shown in 
\cref{fig:msd_NC}.

\begin{figure}[htbp]
    \centering
    \includegraphics[width=.8\textwidth]{figure/MSD/NC_msd.pdf}
    \caption{Here varying the obstacle density the MSD decreases when the obstacle density increases. 
    Another expected result where the active particles have lesser space to move when the arena is 
    filled with more obstacles. When the arena is packed with the number of obstacles as high as 
    1000-1300 obstacles, the active particles will be significantly hindered and unable to move 
    from their starting position, this leads to them having MSD close to zero throughout time $\tau$. 
    The different figures show MSD as a function of obstacle density at different obstacle weight, 
    with $m_{\text{passive}}=\SI{2e-3}{kg}$ in $\textbf{(a)}$, 
    $m_{\text{passive}}=\SI{7e-3}{kg}$ in $\textbf{(b)}$, 
    $m_{\text{passive}}=\SI{12e-3}{kg}$ in $\textbf{(c)}$, 
    and $m_{\text{passive}}=\SI{17e-3}{kg}$ in $\textbf{(d)}$.} 
    \label{fig:msd_NC}
\end{figure}

Here calculating the MSD of the active particles as a function of 
the obstacle density, it shows that the MSD is higher for the cases with low obstacle density and 
decreases as we increase the number of obstacles in the experiment. An expected result 
where less space is available for the active particles to move freely as the number 
of obstacles increases and particle-to-obstacle collisions is more frequent.

\section{Velocity}
To study how the weight of the obstacles affect the behavior of the active particles, 
it is desirable to maximize the interactions between the bugs and the obstacles. 
To achieve this, the experiment with 1200 obstacles and 25 bugs was chosen. The result 
of how the active particles velocity get affected by the weight of the obstacles is shown in 
\cref{fig:velkde_NW1200C25B}.

\begin{figure}[htpb]
    \centering
    \includegraphics[width=.8\textwidth]{figure/velocity/NW1200C25B_vel.pdf}
    \caption{Probability distribution of the active particles velocity depending on the 
    obstacle weight. All the four cases were done using 25 active 
    particles and 1200 obstacles, while the different obstacle weight cases has their 
    probability distribution peaks at velocity close to zero, the distribution for lower 
    obstacles weight has lower peak, slightly shifted to the right and also has a fatter tail. 
    Compare for example $m_{\text{passive}}=\SI{2e-3}{kg}$(blue) and $m_{\text{passive}}=\SI{17e-3}{kg}$(red).}
    \label{fig:velkde_NW1200C25B}
\end{figure}

Here all the cases were done using 25 active particles and with 1200 obstacles, while 
the weight of the obstacles varies in 4 cases, $m_{\text{passive}}=\SI{2e-3}{kg}$, 
$m_{\text{passive}}=\SI{7e-3}{kg}$, $m_{\text{passive}}=\SI{12e-3}{kg}$ and 
$m_{\text{passive}}=\SI{17e-3}{kg}$ producing the four different probability distribution.

What can be observed in \cref{fig:velkde_NW1200C25B} 
is that in all the cases, the probability distributions have similar shapes, peaking around $v\approx\SI{0}{\metre\per\second}$ 
and decrease as the velocity goes higher. Even though the similar shape of these distributions, 
one could see that at $v\approx\SI{0}{\metre\per\second}$, 
the peak of the distribution for lower obstacle weight is slightly lesser than the peak of higher obstacle weight. 
This result suggest that with a high obstacle density($N_{\text{passive}}=1200$), regardless of the obstacle weight, 
the velocity of the active particles will tend to 
be close to zero. However the probability of active particle having zero velocity is highest when using the heaviest obstacle, 
and as the obstacle weight decreases the bugs probability of having $v>\SI{0}{\metre\per\second}$ increases, according 
to the distributions in \cref{fig:velkde_NW1200C25B}.

One of the possible explanation of the similar distribution shapes, even though with different obstacle weights, is that when the arena
is packed with this many obstacles, close to its maximum packing density, the obstacles have less freedom to move 
when being pushed by the active particles. This leaves the active particles no options to move, neither 
by forming a channel nor changing direction to find a new path, leading to them being significantly 
slowed down or getting blocked by the obstacles, similar to a car in a traffic jam. 
Also the number of active particles, in these cases 25, 
may give rise to the situation of overcrowdedness where the active particles work against each other 
by blocking one another or closing already formed channels and thus preventing movements. 
With this high packing density and high number of active particles, regardless of the weight of the 
obstacles, the movement of the active particles will be minimal which leads to velocity close to zero.

Even though the arena hasn't yet reached its maximum capacity in how many obstacles that can be fit in, 
with a high obstacle density such as 1200 obstacles, the steady state behavior will be similar regardless 
of the obstacle weight, which is the active particles being hindered by the obstacles. However, in shorter time periods for example 
10 to 20 seconds, there are noticeable differences between the cases of different obstacle weight, as shown in \cref{fig:snapshot_0020}.

\begin{figure}[htbp]
\centering
\subfloat[$m_{\text{passive}}=\SI{2e-3}{kg}$\label{fig:0W1300C25B_0020}]
{\includegraphics[width=.4\textwidth]{figure/velocity/snapshot/0W1300C25B_snapshot_0020_ipe.PNG}}\quad
\subfloat[$m_{\text{passive}}=\SI{17e-3}{kg}$\label{fig:3W1300C25B_0020}] 
{\includegraphics[width=.4\textwidth]{figure/velocity/snapshot/3W1300C25B_snapshot_0020_ipe.PNG}}
\caption{Snapshot of two experiments both at $\unit[20]{s}$. Both experiments 
was done using 25 active particles and with 1300 obstacles wherein \textbf{(a)} the obstacle 
weigh is $m_{\text{passive}}=\SI{2e-3}{kg}$ each, and in 
\textbf{(b)} $m_{\text{passive}}=\SI{17e-3}{kg}$ each. The trajectories of the active particles in 
red lines shows that at the begining in \textbf{(a)}, the active particles were able to form and/or 
reuse channels for a short time, while in \textbf{(b)} the active particles remain more or less confined
in the same region from where they started.} \label{fig:snapshot_0020}
\end{figure}

Seeing the two cases side-by-side one can notice the obvious differences where the case in \cref{fig:0W1300C25B_0020}, after letting 
the active particles run for $\unit[20]{s}$ they've manage to move the obstacles aside and form 
some noticeable channels. This activity of forming channels and moving through them last for 
some seconds and afterwards the majority of the active particles will return to being 
hindered again, either by pushing a large block of obstacles or by blocking each other and preventing movements.

In \cref{fig:3W1300C25B_0020} the active particles will immediately be hindered by the heavy obstacles right from the start,
and will spend almost all of their time standing still from being blocked and confined by the obstacles.
from pushing the heavy obstacles of $\SI{17e-3}{kg}$ each, leaving all of them 
with velocities close to zero.

Having seen the results above, one would wonder what happens when the packing density and the number 
bugs is lower, will there be more noticeable differences between the different obstacle weight? Below in \cref{fig:velkde_NWNC10B} shows 
probability distributions of 4 different packing densities with 100, 400, 700, 1000 obstacles.

\begin{figure}[htbp]
    \centering
    \includegraphics[width=.8\textwidth]{figure/velocity/NW_vel.pdf}
    \caption{Probability distributions of the active particles velocity at 
    4 different obstacle density. Each figure has 4 distributions 
    corresponding to the different obstacle weight while the number 
    of active particles is fixed to 10. At a high obstacle density 
    case in \textbf{(d)}, the probability $p(v>0)$ for heavier obstacle 
    cases (red, green and orange) is close to zero. The lightest obstacle 
    weight case in blue has its distribution peak significantly lower than 
    the other cases and with a fatter tail, this suggest the situation 
    shown \cref{fig:snapshot_0020} where the active particle are able to 
    move more at this obstacle weight. In \textbf{(c)} show clearly how 
    the different obstacle weight affect the active particles velocity where, 
    the distribution becomes narrower and shift towards zero velocity as the 
    obstacles becomes heavier.} 
    \label{fig:velkde_NWNC10B}
\end{figure}

% \begin{figure}[htbp]
% \centering
% %\captionsetup{justification=centering,margin=2cm}
% \subfloat[100 obstacles\label{fig:velkde_NW100C10B}]
% {\includegraphics[width=.5\textwidth]{figure/velocity/NW/NW100C10B.png}}
% \subfloat[400 obstacles\label{fig:velkde_NW400C10B}]
% {\includegraphics[width=.5\textwidth]{figure/velocity/NW/NW400C10B.png}}\\
% \subfloat[700 obstacles\label{fig:velkde_NW700C10B}]
% {\includegraphics[width=.5\textwidth]{figure/velocity/NW/NW700C10B.png}}
% \subfloat[1000 obstacles\label{fig:velkde_NW1000C10B}]
% {\includegraphics[width=.5\textwidth]{figure/velocity/NW/NW1000C10B.png}}
% \caption{Probability distributions of the active particles velocity at 4 different obstacle density. 
% Each figure has 4 distributions corresponding to the different obstacle weight while the number of active particles is fixed to 10. 
% At a high obstacle density case in \textbf{(d)}, the probability $p(v>0)$ for heavier obstacle cases (red, green and orange) is close to zero. 
% The lightest obstacle weight case in blue has its distribution peak significantly lower than the other cases and with a fatter tail, this 
% suggest the situation shown \cref{fig:snapshot_0020} where the active particle are able to move more at this obstacle weight. 
% In \textbf{(c)} show clearly how the different obstacle weight affect the active particles velocity where, the distribution becomes 
% narrower and shift towards zero velocity as the obstacles becomes heavier.} 
% \label{fig:velkde_NWNC10B}
% \end{figure}

The clearest case that shows how the weight of the obstacles affect the 
velocities of the active particles is the case of 700 obstacles in 
\cref{fig:velkde_NWNC10B}\textbf{(c)}. Here the distribution shift to the left 
and gets narrower as the obstacle weight increases, 
with $\SI{17e-3}{kg}$(red) having highest peak at 
$v\approx\SI{0}{\metre\per\second}$, followed by $\SI{12e-3}{kg}$(green), 
$\SI{7e-3}{kg}$(orange) and $\SI{2e-3}{kg}$(blue). This result show how 
the active particles is being slowed down depending on the obstacle weight, 
where their velocity is lower in an environment of heavier obstacles. 

At lower obstacle densities cases in \cref{fig:velkde_NWNC10B}\textbf{(a)} and 
\textbf{(b)} the range of allowed velocities 
that the bugs can take seems to be larger i.e. there is nonzero 
probability for higher velocity, and there is high probability 
around $v\approx\SI{0.2}{\metre\per\second}$. For these cases 
(\cref{fig:velkde_NWNC10B}\textbf{(a)} and \textbf{(b)}) the overall 
qualitative behavior of the active particles is barely affected 
by the different obstacle weights, since the interactions between 
the passive- and active particles are minimal when the packing density is low. 

So an observation that can be made from the results in 
\cref{fig:velkde_NWNC10B} is that when the packing density 
is too low, there is no clear pattern of how the different 
obstacle weights effect the velocities of the active particles. 
The threshold of ''too low'' packing density is found to be 
about less than \num{500} obstacles. On the other hand, 
when having high packing density where 
$N_{\text{passive}}\gg N_{\text{active}}$, the different 
obstacle weights doesn't have a significant effect on the 
active particles, since at a high packing density, e.g. 
more than \num{1200} obstacles, regardless of the obstacle weight, 
the movement of the active particles will be minimal leaving their 
average velocity close to zero, as shown in \cref{fig:mean_velocity}

\begin{figure}[htpb]
    \centering
    \includegraphics[width=.8\textwidth]{figure/velocity/mixed_vel.pdf}
    \caption{To the left in \textbf{(a)}, the average velocity of 
    the active particles as a function of the obstacle density shows 
    how the average velocity of the active particles decreases as the 
    number of obstacle increases. The different line correspond to different 
    obstacle weight. The average velocity decreases fastest for the heavier 
    obstacles and the lines for $m_{\text{passive}}=\SI{17e-3}{kg}$(red), 
    $m_{\text{passive}}=\SI{12e-3}{kg}$(green), $m_{\text{passive}}=\SI{7e-3}{kg}$(orange) 
    converges at \num{1200} obstacles, while $m_{\text{passive}}=\SI{2e-3}{kg}$(blue) 
    stays at a higher average velocity. The fact that the blue line doesn't converge 
    with the other lines can also be visualized in \textbf{(b)} where the lightest obstacle 
    case(blue) has its distribution peak much lower and also has a fatter tail, at this 
    high packing density.}
    \label{fig:mean_velocity}
\end{figure}

As can be seen in \cref{fig:mean_velocity}\textbf{(a)} 
the active particles average velocity decreases as the number 
of obstacles increases. How fast this velocity decreases does 
depend on the obstacles weight where heavier obstacles 
leads to fastest decrease in velocity, see \cref{fig:mean_velocity}\textbf{(a)}. 
At a high packing density at about \num{1200} obstacles and higher, 
the lines corresponding to the three heaviest obstacle weights, 
red, green and orange, converges at the same low velocity. 
The blue line corresponding to the lightest obstacle weight stays at a higher velocity 
which suggest the scenario mentioned in \cref{fig:snapshot_0020} where the 
active particles, at this light obstacle weight of $m_{\text{passive}}=\SI{2e-3}{kg}$, 
are able to form small channels and move through them for a short period of time, 
eventhough the packing density is as high as \num{1200} obstacles.

Previously, the results show that when having the number of active particles fixed, 
the mean velocity for these active particles decrease towards zero as the number of 
obstacles increases. As for the weight of the obstacles as a parameter, 
the mean velocity of the active particles is higher at a low obstacle weight 
and the velocity is lower when the obstacle weight is high. 

A question that needs to be answered is how does the number of active particles 
affect their mean velocity? Lets answer this by looking at the case where 
the obstacle weight is $m_{\text{passive}}=\SI{7e-3}{kg}$, as shown in \cref{fig:velkde_1WNCNB}.

\begin{figure}[htbp]
    \centering
    \includegraphics[width=.8\textwidth]{figure/velocity/NB_vel.pdf}
    \caption{Probability distributions of the active particles 
    velocity at two different obstacle density. At low obstacle 
    density in \textbf{(a)} the distribution peaks at a lower 
    velocity for the higher number of bugs, while at high obstacle 
    density in \textbf{(b)} all the different distribution peaks at 
    a low velocity close to zero.} 
    \label{fig:velkde_1WNCNB}
\end{figure}

As pointed out earlier, the active particles velocity decreases as 
the number of obstacles increases. This again can be seen in 
\cref{fig:velkde_1WNCNB} where the peaks of the velocity 
distributions shifts towards $v=0$ as the number of obstacles 
increases from $N_{\text{passive}}=100$, in \cref{fig:velkde_1WNCNB}\textbf{(a)}, to 
$N_{\text{passive}}=1300$, in \cref{fig:velkde_1WNCNB}\textbf{(b)}.

When having a closer look at how the number of 
active particles is affecting the behavior, 
such as the case with low obstacle density in 
\cref{fig:velkde_1WNCNB}\textbf{(a)} with 
$N_{\text{passive}}=100$, there is a clear separation 
between the different velocity distributions where the 
distribution for higher number of active particles peaks 
at a lower velocity(red), and the distributions shifts 
towards higher velocity as the number of active particles decreases(blue).
This result shows that the probability of having a lower 
ensemble velocity is higher when the number of active particles is high.

This suggest some sort of crowdedness where the large 
amount of active particles hinders each other from moving 
by creating some sort of congestion, see an example in 
\cref{fig:0W300C25B_congestion}. As the number of obstacles 
increases further as in \cref{fig:velkde_1WNCNB}\textbf{(b)}, 
all the distributions peaks at a velocity close to zero, 
regardless of the number of active particles. Similar to this result 
has been seen before in \cref{fig:mean_velocity} 
where the velocities of the active particles converge towards zero, 
independent of the obstacle weight, as the packing density increases.    


\begin{figure}[htbp]
\centering
\subfloat[$t=\SI{32}{s}$\label{fig:0W300C25B_0032}]
{\includegraphics[width=.3\textwidth]{figure/congestion/0W300C25B_0032_ipe.png}} \quad
\subfloat[$t=\SI{33}{s}$\label{fig:0W300C25B_0033}]
{\includegraphics[width=.3\textwidth]{figure/congestion/0W300C25B_0033_ipe.png}} \quad
\subfloat[$t=\SI{34}{s}$\label{fig:0W300C25B_0034}]
{\includegraphics[width=.3\textwidth]{figure/congestion/0W300C25B_0034_ipe.png}} \quad
\caption{An example showing congestion like behavior, 
in three consecutive frames,  where the active particles hinder each 
other from moving when going in the opposite direction. 
Three particles in the upper right corner and 
two in the upper left corner, the particles being stuck 
are marked with black arrows showing their instantaneous orientation. 
The last frame at $t=\SI{34}{s}$ shows the three particles 
in the upper right corner just escaping out from the congestion 
while the two particles in the upper left corner are still being stuck.}
\label{fig:0W300C25B_congestion}
\end{figure}

This hindering behavior in \cref{fig:0W300C25B_congestion} 
happens more frequently as the number of active particles increases, 
and as a consequence, the ensemble mean velocity of the active 
particles decreases when this happens. 

The differences that was seen when observing the active 
particles velocity as a function of number active particles is that, 
higher number of active particles leads to a velocity 
distribution peak at a lower velocity, 
and the explanation being, because of the stage being 
overpopulated with active particles, they hinder each other 
from moving as in \cref{fig:0W300C25B_congestion} which in 
turn leads to them having low velocity.

Regarding the number of obstacles, there is a strong intuition 
already about how this parameter should affect the active 
particles motion, after having seen the results before in \cref{fig:mean_velocity}. 
The intuition is that the active particles velocity should 
decrease as the number of obstacles increases. 
We confirm this by visualizing the probability density of the active particles 
velocity as a function of the number of obstacels in \cref{fig:velkde_1WNC10B}.


\begin{figure}[htpb]
    \centering
    \includegraphics[width=.8\textwidth]{figure/velocity/NC_vel.pdf}
    \caption{Active particles probability distribution 
        as a function of obstacle density. 
	    The distribution shift to the left towards $v=0$ 
        as the the number of obstacles increases, 
	    an expected result since the active particles 
        has less space to move on when the arena 
	    is filled with many obstacles.
    }
    \label{fig:velkde_1WNC10B}
\end{figure}

Having shortly discussed before, the scenario where 
the obstacle weight do not have any significant 
effect on the active particles when the number of 
obstacles is low. The explanation was that when the 
obstacle density is too sparse, then the interaction 
between active particles and obstacles will be minimal, 
resulting in the different obstacle weight having no 
distinguishable effect on the active particles. 
Though intuition suggest that there certainly are 
differences between a bug colliding with a heavy object
in oppose to a light object. Hitting a heavy object, 
the rod shaped bugs should ''bounce'' off changing 
its direction more rapidly than when hitting a light object. 

To investigate this, the orientational changes of the 
bugs in terms angular velocity are being used, 
to see if their behaviors are expected and how do their 
orientation get affected by different parameters. 
The angular velocity is the change of orientation per 
time unit and can be expressed as

\begin{align}
    \omega =& \dfrac{d\theta}{dt}
\end{align}

where $\omega$ is the angular velocity in unit 
$\si{\radian\per\second}$, $\theta$ is the orientation 
measured in $\si{\radian}$. In \cref{fig:angvelkde_NW600C10B} 
shows how the obstacle weight affect 
the angular velocity of the active particles.  


\begin{figure}[htpb!]
    \centering
    \includegraphics[width=.8\textwidth]
    {figure/velocity/NW_angvel.pdf}
    \caption{Distributions of the active particles angular 
    velocity for different obstacle weights. 
	The four distributions all peaks at $\omega\approx0$, 
    the distribution for lighter obstacle weight(blue) 
	has a higher peak compared to the heavier one(red).
    }
    \label{fig:angvelkde_NW600C10B}
\end{figure}

As can be observed in \cref{fig:angvelkde_NW600C10B}, 
all of the cases has their peaks at 
$d\theta/dt\approx\SI{0}{\radian\per\second}$ 
while for $m_{\text{passive}}=\SI{2e-3}{kg}$(blue) 
the distribution peak is higher than for 
$m_{\text{passive}}=\SI{17e-3}{kg}$(red). 
This suggest that there is a higher probability 
for the active particles to change their orientation 
rapidly in an environment of heavy obstacles, 
more rapidly than with light obstacles. 
This points to what was mentioned earlier about the 
nature of the collisions between active and passive particles. 
When the active particles collide with a heavy obstacle, 
the active particle will ''bounce off'' and change their 
direction more abruptly, similar mechanics as in elastic collisions. 
In contrast with the scenario where the active particles collide with a light obstacle, 
they instead push the obstacle forward for a short distance, then slowly 
changing their direction before going on a new path. 

Finally the angular velocity as a function of obstacle density and number of active particles is 
shown in \cref{fig:angvelkde_1WNCNB}

\begin{figure}[htbp]
    \centering
    \includegraphics[width=.8\textwidth]{figure/velocity/mixed_angvel.pdf}
    \caption{Varying obstacle density in \textbf{(a)} shows that the distribution 
    peak is higher at $\omega\approx0$ as the number of obstacles increases, 
    compare \num{1300} obstacles(red) and \num{100} obstacles(blue). In 
    \textbf{(b)}, varying the number of active particles, shows a higher 
    distibution peak at $\omega\approx0$ for higher number of active particles(red), 
    again a result of congestion. 
    } 
    \label{fig:angvelkde_1WNCNB}
\end{figure}

Somewhat expected result in \cref{fig:angvelkde_1WNCNB}, since it was already observed when 
studying the velocity where the velocity tend to be lower as the number of obstacles and the 
number of active particles increases. Similar pattern holds also for the angular velocity.

%99Regarding how the angular velocity changes as a function of the number of active particles, 
%there is no noticeable trend and most of the cases has their angular velocity distribution peaks around zero. 
%Lastly when analysing the angular velocity of the active particles as a function of the number of 
%obstacles, most of the distributions peaks at zero, see for example in \cref{fig:angvel_2WNC10B}


%\begin{figure}[htpb]
%    \centering
%    %\captionsetup{justification=centering,margin=2cm}
%    \includegraphics[width=.7\textwidth]
%    {figure/angular_velocity/2WNC10B/angular_velocity_2WNC10B_dens_hist.png}
%    \caption{Angular velocity distributions, $\omega$, for experiments with 
%    $m_{\text{passive}}=\SI{12e-3}{kg}$ and $N_{\text{active}}=10$ and each color denotes different 
%    obstacle density.}
%    \label{fig:angvel_2WNC10B}
%\end{figure}

%where the angular velocity of the active particles is shown as a function of the obstacle density, $N_{\text{passive}}$. 
%Although, here in \cref{fig:angvel_2WNC10B}, the angular velocity distributions have peaks at zero for all the different 
%obstacle density cases, there is a noticeable differences in the shape of the distributions where 
%the distribution for high obstacle densities such as 
%$N_{\text{passive}}=1300$(yellow) and $N_{\text{passive}}=900$(blue), is narrower than of lower obstacle densities like
%$N_{\text{passive}}=100$(red) or $N_{\text{passive}}=500$(green). This could be due to the fact that the active particles
%has less freedom to turn i.e. change its orientation when the arena is highly packed with heavy obstacles, which 
%would give rise to the high and narrow peak. In contrast the wider peak from the case of low obstacle density is 
%when the active particle have more space to move around and turn without much constraints from the obstacles.99


\newpage
\section{Collective Behaviours} \label{collective_behaviors}

One of the main focus of this thesis was to design a macroscale 
system in a way to provide the conditions for collective behaviors 
to emerge. To be more specific, we want to observe cooperative 
behaviors such as, active particles forming and reusing channels. 
Though this particular behavior is not the only cooperative behavior 
since any other behaviors/interactions between the active particles 
that are mutually beneficial can be classified as cooperative behavior. 
In our case, it is natural to define the goal for the active 
particles as maximizing their movements.

In the experiments, there were some frequently occurring behaviors 
with the bugs that was difficult to quantify. The easiest way 
was to do a qualitative observation by watching the recorded 
footage of the experiments. Some of these behaviors occurs more 
frequent than others and may seem trivial. In this section, 
different observed behaviors will be presented. 

One behavior that happens in almost all the experiments is where the 
bugs become stuck at the boundary edges of the arena, see \cref{fig:boundary}. 
This was a big issue with the two previous designs of the border 
where we first had a rectangular border with $90^\circ$ 
corners(right boundary in \cref{fig:boundary} without the rounded corners) 
and the bugs kept getting stuck at these sharp corners. 
The second iteration had rounded corners which remove the 
issue of the first design but still have the problem of which the bugs 
would spend most of the time traversing the edges of the stage 
instead of interacting with the obstacles inside the stage. 
The current design which is the cloudy border would minimize 
these two previous problems, but the getting ''stuck'' behavior 
still occur, but for different reasons. For the first two border designs, 
the bugs were getting stuck mainly because of the chirality of there 
movements in combination with the border design of straight edges with 
rounded or sharp corners. The chirality of these robots was a 
random property that could not be controlled in before hand, 
also given the fact that the robots are suppose to be toys for children, 
this unexpected property was no surprise. 

With the current ''cloudy'' border(left model in \cref{fig:boundary}) 
the bugs are instead getting stuck when they are going against each 
other in the opposite direction and they are in some sort 
of equilibrium where the forces from each bugs in different 
directions balance out and the group of bugs stay still 
for some seconds until the equilibrium breaks and each bugs go 
their own way again. Of course the chirality of the bugs 
and the stage design still plays a role in this but it is not 
obvious as in the previous two border designs. Moreover, the 
number of the bugs does play a role in this behavior where 
more number of bugs will make this congestion behavior happen 
more frequently, see for example the velocity distribution in 
\cref{fig:velkde_1WNCNB}.

Some example of bugs getting stuck is shown in \cref{fig:BB}.

\begin{figure}[htpb!]
    \centering
    \subfloat[2 bugs\label{fig:0W300C10B_2BB}]
    {\includegraphics[width=.2\textwidth]{figure/BB/2BB/0W300C10B/0W300C10B_2BB_post.png}} \quad
    \subfloat[3 bugs\label{fig:0W100C20B_3BB}]
    {\includegraphics[width=.2\textwidth]{figure/BB/3BB/0W100C20B/0W100C20B_3BB_post.png}} \quad
    \subfloat[4 bugs\label{fig:0W300C25B_4BB}]
    {\includegraphics[width=.2\textwidth]{figure/BB/4BB/0W300C25B/0W300C25B_4BB_post.png}} \quad
    \subfloat[5 bugs\label{fig:0W400C25B_5BB}]
    {\includegraphics[width=.2\textwidth]{figure/BB/5BB/0W400C25B/0W400C25B_5BB_post.png}}\\
    \subfloat[6 bugs\label{fig:1W200C25B_6BB}]
    {\includegraphics[width=.2\textwidth]{figure/BB/6BB/1W200C25B/1W200C25B_6BB_post.png}} \quad
    \subfloat[7 bugs\label{fig:0W1200C15B_7BB}]
    {\includegraphics[width=.2\textwidth]{figure/BB/7BB/0W1200C15B/0W1200C15B_7BB_post.png}} \quad
    \subfloat[8 bugs\label{fig:1W800C25B_8BB}]
    {\includegraphics[width=.2\textwidth]{figure/BB/8BB/1W800C25B/1W800C25B_8BB_post.png}} \quad
    \subfloat[9 bugs\label{fig:2W300C25B_9BB}]
    {\includegraphics[width=.2\textwidth]{figure/BB/9BB/2W300C25B/2W300C25B_9BB_post.png}}

    \caption{Example of situations where the bugs become 
    stuck when coming in contact with one another and the 
    border of the arena. The direction of where the bugs 
    are pushing is marked with black arrows. Each figure 
    shows different number of bugs that are involved, 
    2 bugs in \textbf{(a)}, 3 bugs in \textbf{(b)}, 
    4 bugs in \textbf{(c)}, 5 bugs in \textbf{(d)}, 
    6 bugs in \textbf{(e)}, 7 bugs in \textbf{(f)}, 
    8 bugs in \textbf{(g)} and 9 bugs in \textbf{(h)}} 
    \label{fig:BB}
\end{figure}

The situation shown in \cref{fig:BB} is an example of the active 
particles working against each other by preventing each other from moving. 
This behavior can be categorised as ''Anti Cooperative Behaviour'' 
i.e. some collective behavior that minimize the active particles 
movement, as oppose to Cooperative behavior where the bugs 
help each other to maximize the groups movements.

There are many scenarios where cooperative behavior occurs where 
the bugs assist each other to move in various ways, some easier 
to observed than others. Here some cases of cooperative behavior 
will be presented. With regard of the restrictions of this format, 
only the cases that can be clearly visualized will be presented. 
The first example is shown in \cref{fig:CBPO}. 

\begin{figure}[htpb!]
\centering
\captionsetup[subfigure]{labelformat=empty}
\subfloat[$t_0$\label{fig:0W700C25B_frame1}]
{\includegraphics[width=.2\textwidth]{figure/CBPO/0W700C25B/0W700C25B_CBPO_post_frame1.png}} \quad
\subfloat[$t_1$\label{fig:0W700C25B_frame2}]
{\includegraphics[width=.2\textwidth]{figure/CBPO/0W700C25B/0W700C25B_CBPO_post_frame2.png}} \quad
\subfloat[$t_2$\label{fig:0W700C25B_frame3}]
{\includegraphics[width=.2\textwidth]{figure/CBPO/0W700C25B/0W700C25B_CBPO_post_frame3.png}} \quad
\subfloat[$t_3$\label{fig:0W700C25B_frame4}]
{\includegraphics[width=.2\textwidth]{figure/CBPO/0W700C25B/0W700C25B_CBPO_post_frame4.png}} \\
\subfloat[$t_0$\label{fig:0W800C20B_frame1}]
{\includegraphics[width=.2\textwidth]{figure/CBPO/0W800C20B/0W800C20B_CBPO_post_frame1.png}} \quad
\subfloat[$t_1$\label{fig:0W800C20B_frame2}]
{\includegraphics[width=.2\textwidth]{figure/CBPO/0W800C20B/0W800C20B_CBPO_post_frame2.png}} \quad
\subfloat[$t_2$\label{fig:0W800C20B_frame3}]
{\includegraphics[width=.2\textwidth]{figure/CBPO/0W800C20B/0W800C20B_CBPO_post_frame3.png}} \quad
\subfloat[$t_3$\label{fig:0W800C20B_frame4}]
{\includegraphics[width=.2\textwidth]{figure/CBPO/0W800C20B/0W800C20B_CBPO_post_frame4.png}} \\
\subfloat[$t_0$\label{fig:0W900C10B_frame1}]
{\includegraphics[width=.2\textwidth]{figure/CBPO/0W900C10B/0W900C10B_CBPO_post_frame1.png}} \quad
\subfloat[$t_1$\label{fig:0W900C10B_frame2}]
{\includegraphics[width=.2\textwidth]{figure/CBPO/0W900C10B/0W900C10B_CBPO_post_frame2.png}} \quad
\subfloat[$t_2$\label{fig:0W900C10B_frame3}]
{\includegraphics[width=.2\textwidth]{figure/CBPO/0W900C10B/0W900C10B_CBPO_post_frame3.png}} \quad
\subfloat[$t_3$\label{fig:0W900C10B_frame4}]
{\includegraphics[width=.2\textwidth]{figure/CBPO/0W900C10B/0W900C10B_CBPO_post_frame4.png}}

\caption{Three different examples of cooperative behavior 
showing how the bugs assist one another to push away obstacles. 
It would take much longer time without other bugs helping. 
The direction of the bugs is marked with black arrows.}
\label{fig:CBPO}
\end{figure}


Here the bugs assist each other to push away some 
obstacles from the border of the arena, and thus 
making way for other bugs to move freely at the edges. 
The observation in \cref{fig:CBPO} is taken from three 
different experiments where in each experiment four consecutive 
frames are taken showing how the bugs move. Note that in these 
examples shown, the assistance of other bugs that were not initially pushing 
the obstacles do make a significant difference in moving them, 
either by speeding up the process or clearing out a bigger area of obstacles.

Next example is when the bugs are pushing onto each other 
preventing the scenario where some bugs are being blocked 
and confined by obstacles.

\begin{figure}[htpb!]
\centering
\captionsetup[subfigure]{labelformat=empty}
\subfloat[$t_0$\label{fig:1W800C25B_frame1}]
{\includegraphics[width=.2\textwidth]{figure/CBP/1W800C25B/1W800C25B_CBP_post_frame1.png}} \quad
\subfloat[$t_1$\label{fig:1W800C25B_frame2}]
{\includegraphics[width=.2\textwidth]{figure/CBP/1W800C25B/1W800C25B_CBP_post_frame2.png}} \quad
\subfloat[$t_2$\label{fig:1W800C25B_frame3}]
{\includegraphics[width=.2\textwidth]{figure/CBP/1W800C25B/1W800C25B_CBP_post_frame3.png}} \quad
\subfloat[$t_3$\label{fig:1W800C25B_frame4}]
{\includegraphics[width=.2\textwidth]{figure/CBP/1W800C25B/1W800C25B_CBP_post_frame4.png}} \\
\subfloat[$t_0$\label{fig:1W900C10B_frame1}]
{\includegraphics[width=.2\textwidth]{figure/CBP/1W900C10B/1W900C10B_CBP_post_frame1.png}} \quad
\subfloat[$t_1$\label{fig:1W900C10B_frame2}]
{\includegraphics[width=.2\textwidth]{figure/CBP/1W900C10B/1W900C10B_CBP_post_frame2.png}} \quad
\subfloat[$t_2$\label{fig:1W900C10B_frame3}]
{\includegraphics[width=.2\textwidth]{figure/CBP/1W900C10B/1W900C10B_CBP_post_frame3.png}} \quad
\subfloat[$t_3$\label{fig:1W900C10B_frame4}]
{\includegraphics[width=.2\textwidth]{figure/CBP/1W900C10B/1W900C10B_CBP_post_frame4.png}} \\
\subfloat[$t_0$\label{fig:1W900C5B_frame1}]
{\includegraphics[width=.2\textwidth]{figure/CBP/1W900C5B/1W900C5B_CBP_post_frame1.png}} \quad
\subfloat[$t_1$\label{fig:1W900C5B_frame2}]
{\includegraphics[width=.2\textwidth]{figure/CBP/1W900C5B/1W900C5B_CBP_post_frame2.png}} \quad
\subfloat[$t_2$\label{fig:1W900C5B_frame3}]
{\includegraphics[width=.2\textwidth]{figure/CBP/1W900C5B/1W900C5B_CBP_post_frame3.png}} \quad
\subfloat[$t_3$\label{fig:1W900C5B_frame4}]
{\includegraphics[width=.2\textwidth]{figure/CBP/1W900C5B/1W900C5B_CBP_post_frame4.png}}

\caption{Examples of cooperative behavior where one 
bug push another bug that is being blocked, to move forward. 
When facing an obstacle one single bug might not be strong 
enough to move the obstacles in front, but when another bug join, 
they are able to together push the obstacle away making channels for other bugs.
} 
\label{fig:CBP}
\end{figure}


As for the case of cooperative behavior shown i \cref{fig:CBP}, 
where a push bug push another bug, the outcome where the pushing 
bug successfully push another bug from its unmovable position 
doesn't always happen. When the obstacle weight is too high and 
if there are too many obstacle in front of the bugs, both the bugs 
will become stuck. With that said, when two bugs do collide 
they tend to change each others movement in some way and thus 
destroy whatever equilibrium they were in, which will create more movements.

The motivation of categorising the above behaviors in \cref{fig:CBPO,fig:CBP} as 
''Cooperative behaviors'' can be put as follows. Since 
the aim of this project is to study the collective behaviors of these agents, 
and to do so, these agents have to ''behave'' in some way, and behaving in 
this particular case is almost synonymous with moving. Moving particles thus give 
interesting and non-trivial data for analysis, compared to a particle standing still 
from being blocked. So any action, from a particle onto another one, that increase the 
ensemble displacement will be considered as cooperative behavior.


Now the next reoccurring behavior, which is shown 
in \cref{fig:CMC}, is when the bugs create a circular 
shaped channel and keep reusing it, circling inside for 
many lapses. 


\begin{figure}[htpb!]
\centering
\subfloat[7 lapses\label{fig:CMC_0W1200C15B_a}]
{\includegraphics[width=.30\textwidth]{figure/CMC/0W1200C15B/0W1200C15B_CMC_post.png}}\quad
\subfloat[3 lapses\label{fig:CMC_1W800C10B_b}]
{\includegraphics[width=.30\textwidth]{figure/CMC/1W800C10B/1W800C10B_CMC_post.png}}\\
\subfloat[2 lapses\label{fig:CMC_1W800C15B_c}]
{\includegraphics[width=.30\textwidth]{figure/CMC/1W800C15B/1W800C15B_CMC_post.png}}\quad
\subfloat[4 lapses\label{fig:CMC_1W1000C20B_d}]
{\includegraphics[width=.30\textwidth]{figure/CMC/1W1000C20B/1W1000C20B_CMC_post.png}}
\caption{Examples where the bugs are circling inside 
a circle shaped channel for many lapses. The bugs might 
end up in this situation coming from other channels, 
and due to its chirality they might circle around for some 
lapse before moving on through nearby channels. In some cases 
while the bugs are circling, other bugs might push and move 
the obstacles adding walls to this circular shaped channels making 
them more stable. This leads to the bugs inside, the fully closed 
circular channel, traversing around for many lapses, 
as seen in \textbf{(a)} and \textbf{(b)}.} 
\label{fig:CMC}
\end{figure}

This circling behavior is presented in the four examples, 
\cref{fig:CMC_0W1200C15B_a,fig:CMC_1W800C10B_b,fig:CMC_1W800C15B_c,fig:CMC_1W1000C20B_d}, 
with the trajectories of the bugs marked in red and the number 
of lapses of which the bugs are circling around. 

This type of ''local'' channel i.e. not expanding over a large 
area of the arena, tend to be more stable than the channels 
that extend over a larger area. The reason for this is that 
for the latter type of channel, the likelihood of another bug 
cutting through and destroy the channel is higher. These circular 
shaped channels are mostly formed between rigid boundary from the stage(\cref{fig:boundary}) 
and obstacles, this means that a portion of the channel will be an 
immovable part supported by the rigid bounday, and the movable part formed 
by the obstacles. This will decrease the chance of channels being 
destroyed since it is partly immovable.

Now the duration of the bugs circling in these channels will 
depend on how much of the channel consist of the rigid boundary, 
which is immovable by the bugs. In \cref{fig:CMC_0W1200C15B_a}, 
the bug manage to stay in the channel for 7 lapses compared to 
the case in \cref{fig:CMC_1W800C15B_c}, which is only 2 lapses. 
One could immediately see the differences in these two channels 
where in the first case a big segment of the channel constitute 
of the rigid boundary made by wood, roughly two-thirds of the channel. 
Whereas the latter case where the bug only circle for 2 lapses, 
this channel constitute mostly of obstacles and empty spaces, 
roughly two-third of the channel being movable and one-third immovable, 
this will increase the likelihood of the bug destroying or escaping the channel. 

Generally the channels tend to be quite unstable, especially the ones 
that cover a large area of the stage consisting only of obstacles and 
no rigid boundaries. Every time a bug uses a channel it moves 
the obstacles slightly, and after many usages, the channel is destroyed. 
A channel is destroyed faster when using light obstacles and high number of bugs.

An example of the type of channel that stretches across the stage 
is shown in \cref{fig:CM_0W900C10B}.

\begin{figure}[htpb!]
\centering
\subfloat{\includegraphics[width=.7\textwidth]{figure/CM/0W900C10B/0W900C10B_CM_post_1.png}}
\caption{A bug propagating through a long channel that stretches over the stage, 
starting from the upper left corner going to the lower right corner.
This experiment was done with $N_{\text{active}}=10$, $N_{\text{passive}}=900$ and 
$m_{\text{passive}}=\SI{2e-3}{kg}$. The trajectory of where the bug traverses along the 
channel is marked with a dashed red line, the start of the trajectory is marked as $\vdash$ and the 
end with $\rightarrow$. The bug that traverses this trajectory is marked with a black dot.} 
\label{fig:CM_0W900C10B}
\end{figure}

Again, these type of channel shown in \cref{fig:CM_0W900C10B} 
tend to be unstable with reasons mentioned before. 
Compared to the circular shaped channels shown in \cref{fig:CMC}, these long type 
are quite exposed to other bugs destroying it, by either pushing away 
the obstacles that make up the walls of the channels or moving some 
obstacles into the channel and thus blocking the channel. 

With only 10 bugs in this experiment, in \cref{fig:CM_0W900C10B}, 
it is relatively easy to spot when a channel is being created, 
reuse and destroyed. When the number of bugs is higher than this, the 
stage gets somewhat cluttered which makes it harder to spot a channel 
and also the high number of bugs will raise the likelihood of a channel being destroyed.

With only one bug, it is easier to track its behavior, as seen in \cref{fig:CM_0W1000C1B}.

\begin{figure}[htpb!]
\centering
\subfloat[$t_{\text{duration}}=\SI{9}{\second}$\label{fig:CM_0W1000C1B_1}]
{\includegraphics[width=.45\textwidth]{figure/CM/0W1000C1B/0W1000C1B_CM_post_1.png}}\quad
\subfloat[$t_{\text{duration}}=\SI{11}{\second}$\label{fig:CM_0W1000C1B_2}]
{\includegraphics[width=.45\textwidth]{figure/CM/0W1000C1B/0W1000C1B_CM_post_2.png}}\\
\subfloat[$t_{\text{duration}}=\SI{4}{\second}$\label{fig:CM_0W1000C1B_3}]
{\includegraphics[width=.45\textwidth]{figure/CM/0W1000C1B/0W1000C1B_CM_post_3.png}}\quad
\subfloat[$t_{\text{duration}}=\SI{11}{\second}$\label{fig:CM_0W1000C1B_4}]
{\includegraphics[width=.45\textwidth]{figure/CM/0W1000C1B/0W1000C1B_CM_post_4.png}}
\caption{Four occurrences in the same experiment where a bug is reusing preexisting channels. 
Depending on the length of the channel, it takes different time for the bug to go through 
the whole channel ranging from $t_{\text{duration}}=\SI{4}{\second}$ in \textbf{(c)} to 
$t_{\text{duration}}=\SI{11}{\second}$ in \textbf{(b)} and \textbf{(d)}.
Since there is only one bug in this experiment, this bug both create and reuse its own channel.
This experiment was done with $N_{\text{active}}=1$, $N_{\text{passive}}=1000$ and 
$m_{\text{passive}}=\SI{2e-3}{kg}$.} 
\label{fig:CM_0W1000C1B}
\end{figure}

In this one bug case it is easy to spot the 
moment when the bug is using a channel, 
and some part of the channel last relatively 
long since there is no other bug that can 
destroy the channel. The channels that are 
formed in the experiments with low number of bugs 
and high obstacle density tend to be longer in distance 
and more stable i.e. stay intact for a longer period of time. 
This can be visualized in \cref{fig:CM_0W1000C1B_1,fig:CM_0W1000C1B_2,fig:CM_0W1000C1B_3}, 
where the overall appearance of the stage, with the preexisting channels, looks 
close to identical. In the one bug example, in \cref{fig:CM_0W1000C1B} the duration 
for with the bug to reuse the different channels spans from $t_{\text{duration}}=\SI{4}{\second}$ 
to $t_{\text{duration}}=\SI{11}{\second}$, which are some of the longest duration observed 
amongst the different experiments regarding channel behaviors.

Another type of situation where several bugs reuse the same 
channel is presented in \cref{fig:CM_0W900C25B}.

\begin{figure}[htpb!]
\centering
\subfloat[\label{fig:CM_0W900C25B_1}]
{\includegraphics[width=.45\textwidth]{figure/CM/0W900C25B/0W900C25B_CM_post_1.png}}\quad
\subfloat[\label{fig:CM_0W900C25B_2}]
{\includegraphics[width=.45\textwidth]{figure/CM/0W900C25B/0W900C25B_CM_post_2.png}}
\caption{Situation where several bugs traverse through the same channel, 
7 bugs in \textbf{(a)} and 2 bugs in \textbf{(b)}.
This experiment was done with $N_{\text{active}}=25$, $N_{\text{passive}}=900$ and 
$m_{\text{passive}}=\SI{2e-3}{kg}$.} 
\label{fig:CM_0W900C25B}
\end{figure}


With the one bug example in \cref{fig:CM_0W1000C1B}, its channelling behavior 
can also be observed by looking at the time 
evolution of its velocity $v$, reported in \cref{fig:CM_0W1000C1B_vt}.

\begin{figure}[htbp]
    \centering
    \includegraphics[width=.7\textwidth]{figure/velocity/0W1000C1B_vt.pdf}
    \caption{Velocity-time graph showing how the velocity of 
    the bug in \cref{fig:CM_0W1000C1B}, evolves through time. 
    The red segments of the curve marked with \textbf{(a)}, \textbf{(b)}, 
    \textbf{(c)} and \textbf{(d)} is the period when the bug is traversing 
    through the channels shown in \cref{fig:CM_0W1000C1B}. At the channel-using 
    segments (red lines), the bugs velocity oscillate around a higher velocity 
    than the blue segments, where the bug is instead forming channels. In the 
    channel-using segment \textbf{(d)} there is a short velocity dip in the middle. 
    Eventhough this being a channel-using segment, the bug collide against the walls 
    of the channel and push for a short period of time before returning to going through 
    the already existing channel again.} 
    \label{fig:CM_0W1000C1B_vt}
\end{figure}

This figure shows how the bugs velocity vary depending on its behavior, whether 
the bug is reusing an already existing channel or the bug is forming a new channel. 
When the bug is reusing a channel, its velocity oscillates around a relatively 
higher velocity(red segments in \cref{fig:CM_0W1000C1B_vt}) 
than when it is forming a channel(blue segments in \cref{fig:CM_0W1000C1B_vt}). 
This velocity differences is due to the simple fact that, when the bug 
is reusing a channel, it can move throgh obstacle-free spaces unhindered 
and is able to maintain its high speed throughout the stretch of the channel. 
On the other hand, when the bug is forming a channel, it instead spending 
most of it time pushing obstacles trying to form new channels, which will 
inevitably slow its speed down.   

% In this figure it can seen that the velocity is fluctuating between minima- and maxima values. 
% As expected the minimas marked  \textbf{(a)} and \textbf{(d)} connects to the situations 
% where the bug is being blocked by trying to penetrate through the obstacles. The maximas marked in 
% \textbf{(b)} and \textbf{(c)} is the scenario where the bug moves unhindered through a channel. 
% Snapshots of these four points is shown in \cref{fig:0W1000C1B_vt_refer}

% \begin{figure}[htbp!]
% \centering
% \subfloat[\label{fig:0W1000C1B_vt_refer_a}]
% {\includegraphics[width=.30\textwidth]{figure/CM/0W1000C1B/0W1000C1B_CM_post_a.png}}\quad
% \subfloat[\label{fig:0W1000C1B_vt_refer_b}]
% {\includegraphics[width=.30\textwidth]{figure/CM/0W1000C1B/0W1000C1B_CM_post_b.png}}\\
% \subfloat[\label{fig:0W1000C1B_vt_refer_c}]
% {\includegraphics[width=.30\textwidth]{figure/CM/0W1000C1B/0W1000C1B_CM_post_c.png}}\quad
% \subfloat[\label{fig:0W1000C1B_vt_refer_d}]
% {\includegraphics[width=.30\textwidth]{figure/CM/0W1000C1B/0W1000C1B_CM_post_d.png}}
% \caption{This figure shows four snapshot of the experiment with
% $N_{\text{active}}=1$, $N_{\text{passive}}=1000$ and 
% $m_{\text{passive}}=\SI{2e-3}{kg}$. Each of these four figures represent the marked 
% maximas and minimas in \cref{fig:CM_0W1000C1B_vt}} 
% \label{fig:0W1000C1B_vt_refer}
% \end{figure}

% As expected the points with low velocity, \textbf{(a)} and \textbf{(d)} in 
% \cref{fig:CM_0W1000C1B_vt}, relates to when the bug is being blocked by obstacles, which are the scenarios 
% in \cref{fig:0W1000C1B_vt_refer_a,fig:0W1000C1B_vt_refer_d}. The 
% two velocity maximas \textbf{(b)} and \textbf{(c)} shows instead that the bug is 
% traversing through a channel unhindered which \cref{fig:0W1000C1B_vt_refer_b,fig:0W1000C1B_vt_refer_c}
% confirms.

