\chapter{Introduction} 

%Maybe start differently instead of With many particle systems...
With many particle systems, one of the most interesting aspects is the complex 
collective behaviors that emerge. In nature, this behavior can be observed in, 
schools of fish moving as a unit, flocks of birds flying uniformly as group, 
herd of land animals and even human crowds. These phenomena are fascinating 
to observe and study, because the mechanisms behind the collective behaviors are 
far from obvious.

Collective behaviors have been studied in great detail for the last decades. 
Early on, self-propelled particles were studied to model the swarm behavior of 
animals at the macroscale\cite{bechinger2016active}. Reynolds introduced in 1987 
a ''Boids model'' to simulate noncolliding aggregate motion, such as that of 
flocks of birds, heards of land animals, and schools of fish within computer 
graphics applications\cite{reynolds1987flocks}. In 1995, Vicsek and co-authors 
introduced the ''Vicsek model'' where the swarm behavior is modelled by active 
particles that are driven with a constant absolute velocity and they tend to 
align with the average direction of motion of the particles in their 
neighbourhood\cite{vicsek1995novel}. The Vicsek model was the first model to 
look at collective motion as a noise-induced phase transition. Then later on, 
several other models have been introduced aiming to study and explain the 
properties of collective behaviors\cite{chate2008modeling,grossmann2013self,
barberis2016large,mijalkov2016engineering, volpe2016effective,matsui2017noise,
cambui2017finite}. Experimental studies have also been done on systems with 
complex collective behaviors\cite{czirok1996formation, palacci2013living, 
theurkauff2012dynamic, ginot2015nonequilibrium, morin2017distortion}.
The collective behavior in swarming systems turns out to occur in many 
different scales, and furthermore the behaviors are robust and universal 
e.g. the animals share group-level properties, irrespective of the type of 
animals in the group\cite{buhl2006disorder}.

With the ''Vicsek model'', active particles were introduced to model the 
swarm behavior. The term ''active'' refers to the ability of the individual 
particles to move actively by gaining energy from the 
environment\cite{romanczuk2012active}. Examples of such systems range from 
microsystems such as brownian motors\cite{reimann2002brownian} and 
motile cells\cite{friedrich2007chemotaxis,selmeczi2008cell,boedeker2010quantitative}, 
to macroscopic animals\cite{kareiva1983analyzing,komin2004random} and also 
artificial self-propelled particles\cite{paxton2004catalytic,howse2007self}. 
Active particles are able to propel themselves and perform active motion due 
to an internal driving. This could have been caused by different factors such 
as biological activity or non-equilibrium dynamics in artificial driven 
systems\cite{romanczuk2012active}. This ability of self-propulsion is a 
common feature in microorganisms\cite{lauga2009hydrodynamics, cates2012diffusive, poon2013clarkia}, 
which allows the organisms for a more efficient way to search for nutrients or 
avoid toxic substances\cite{viswanathan2011physics}. In contrast to the active particles, 
the motion of passive particles is the standard dynamical behavior of particles 
suspended in a medium, and is driven by equilibrium thermal fluctuations, due to 
random collisions with the surrounding fluid molecules\cite{babivc2005colloids}. 

Some of the most generic models used to describe active particles systems can be considered 
as an extension of well known concepts in physics such as Brownian motion. Brownian motion, 
being purely physical in origin and having a central role in the foundation of thermodynamics and 
statistical physics, is now a major interdisciplinary research topic. The concept of self-propulsion 
is often studied in the framework of active Brownian particles. In many publications 
the term ''Active Brownian Particles'' is used to refer to self-propelled particles 
far from equilibrium (see e.g \cite{schweitzer1998complex,ebeling1999active,erdmann2003collective,
schweitzer2003brownian,romanczuk2008beyond}). Here, we will refer to ''Active Brownian Particles'' 
as Brownian particles performing active Brownian motion.

In recent years, active Brownian motion has attracted the interest of the biology 
and physics communities\cite{ebbens2010pursuit, poon2013clarkia}. Several types of 
microscopic biological systems perform active Brownian motion. 
%; a paradigmatic example is the swimming behavior of bacteria such as Escherichia coli\cite{volpe2014simulation}. 
Understanding their motion can provide insight into out-of-equilibrium phenomena\cite{volpe2014simulation} 
and lead to the development of novel strategies for designing smart devices and materials\cite{bechinger2016active}. 
The possibility of designing and using active particles in real world application is immense, 
ranging from the targeted delivery of drugs, biomarkers, or contrast agents in 
health care applications\cite{nelson2010microrobots, wang2012nano, patra2013intelligent, abdelmohsen2014micro},
to the autonomous depollution of water and soils, 
climate changes, or chemical terroristic attacks in sustainability and security applications\cite{gao2014environmental}. 

Active particles provide great hope in adressing the many challenges of our modern societies and 
a significant and growing effort has been pushed to advancing this field and to explore its applications in a 
diverse set of disciplines\cite{bechinger2016active}; 
in statistical physics\cite{ramaswamy2010mechanics}, biology\cite{viswanathan2011physics}, 
robotics\cite{brambilla2013swarm}, social transport\cite{helbing2001traffic}, 
soft matter\cite{marchetti2013hydrodynamics} and biomedicine\cite{wang2012nano}. 
The potential applications can be built around the core functionalities of active 
Brownian particles, with transport, sensing and manipulation, which can lead 
to smart designs of nanomachines and micromachines that can perform tasks in an autonomous, 
targeted and selective way.

%Active particles provide great hope in addressing 
%challenges of our modern societies such as personalized health care, environmental 
%sustainability and security\cite{nelson2010microrobots, wang2012nano, patra2013intelligent, 
%gao2014environmental, ebbens2016active}. 

This property of universality and scalabilty of collective behavior will be the main focus 
of this thesis, especially on systems in macroscale. And efforts will be made in trying to answer 
whether a model describing the behavior of active Brownian particles is scalable i.e. valid 
in different scales.



\section{Related works}
A study was done in \citeyear{nilsson2017metastable} by \citeauthor{nilsson2017metastable} using a simulation of active particles with 
short-range aligning interactions\cite{nilsson2017metastable}. This model was studied numerically as a function of 
orientational noise parameter. The simulation was done with and without the presence of passive particles, 
and it was shown that, with the presence of passive particles, the active particles transition from a diffusive 
state, at high noise levels, to a state of which they can propagate unhindered along a network of metastable channels 
as the noise level is decreased. 

In this model, the position $\bm{x}_n$ and direction $\theta_n$, of particle $n$ 
is updated each timestep $t=0, 1, 2, \ldots$ according to 

\begin{align}
    \begin{cases}
        \bm{x}_n(t+1)       &= \bm{x}_n(t) + \bm{v}_n(t+1) \\
        \theta_n(t+1)       &=  \theta_n(t) + T_n + \xi
    \end{cases}
\end{align}

The particles are hard spheres with a constant speed of $|\bm{v}_n|\equiv v$, $\xi$ is a white-noise 
term which is uniformly distributed in the interval $\left[-\eta/2, \eta/2\right]$, and $T_n$ is a 
torque term. The torque term describe the torque exerted on particle $n$ by all other particles, active 
and passive, and is expressed as follows

\begin{align}
    T_n &=  T_0\sum_{i\neq n}\dfrac{\bm{\hat{v}}_n\cdot\bm{\hat{r}}_{ni}}{r^2_{ni}}
            \bm{\hat{v}}_n\times\bm{\hat{r}}_{ni}\cdot\bm{\hat{e}}_z - 
            T_0\sum_{m}\dfrac{\bm{\hat{v}}_n\cdot\bm{\hat{r}}_{nm}}{r^2_{nm}}
            \bm{\hat{v}}_n\times\bm{\hat{r}}_{nm}\cdot\bm{\hat{e}}_z\text{ for } r_{ni}, r_{nm} < r_c,
            \label{eq:torque}
\end{align}

The first term in \cref{eq:torque} describes the torque exerted on the active particle $n$ by all other active particles. The second 
term is the torque exerted on the same active particle $n$ by all the passive particles $m$ where $m = 1,\ldots,M$.
For more detail on the theory, see \cite{nilsson2017metastable}. A simulation of this model was done using 20 active particles 
and 900 passive particles, with various noise level the result is shown in \cref{fig:simon_noise}

\begin{figure}[htbp]
\centering
%\captionsetup{justification=centering,margin=2cm}
\subfloat[$\eta=2\pi$\label{fig:simon_noise_2pi}]
{\includegraphics[width=.3\textwidth]{figure/external_sources/simon/siom_a4.PNG}}
\subfloat[$\eta=\pi$\label{fig:simon_noise_pi}]
{\includegraphics[width=.3\textwidth]{figure/external_sources/simon/simon_b4.PNG}}\\
\subfloat[$\eta=0.5\pi$\label{fig:simon_noise_05pi}]
{\includegraphics[width=.3\textwidth]{figure/external_sources/simon/simon_c4.PNG}}
\subfloat[$\eta=0.03\pi$\label{fig:simon_noise_003pi}]
{\includegraphics[width=.3\textwidth]{figure/external_sources/simon/simon_d4.PNG}}

\caption{Taken from \cite{nilsson2017metastable}, a snapshots at timestep $t = \num{100000}$ of a 
simulation with 20 active particles(red dots) and 900 
passive particles(white dots). The behavior of the active particles is captured at 4 different directional 
noise levels, $\eta=2\pi$ in \textbf{(a)}, $\eta=\pi$ in \textbf{(b)}, $\eta=0.5\pi$ in \textbf{(c)} 
and $\eta=0.03\pi$ in \textbf{(d)}.} \label{fig:simon_noise}
\end{figure}

At a high noise level with $\eta=2\pi$ in \cref{fig:simon_noise_2pi}, the active particles motion is significantly 
restricted and perform essentially a Brownian diffusive motion while being confined in small pockets surrounded by passive particles. 
On the other hand when the noise level is decreased as in \cref{fig:simon_noise_003pi}, the active particles is 
able to move freely forming channels to which they can propagate through and resuse.

This behavior was also measured quantitatively in terms of the mean square displacement (MSD), which 
is a tool to characterise a particles movements, whether a particle spreads randomly due to diffusion, or 
if there is a force contributing to the movements. The MSD is shown in \cref{fig:simon_msd}.

\begin{figure}[htp]
    \centering
    %\captionsetup{justification=centering,margin=2cm}
    \includegraphics[scale=0.7]{figure/external_sources/simon/simon_msd.PNG}
    \caption{From \cite{nilsson2017metastable} where the MSD 
    of active particles in the presence of passive particles is plotted as a function of the directional
    noise level $(\eta)$, in the conditions shown in \cref{fig:simon_noise}.}
    \label{fig:simon_msd}
\end{figure}

The MSD shows that at a high noise level where $\eta=2\pi$, the motion of the active particles 
is significantly hindered, resulting in the MSD curve having a slope $\leq1$ at all times(green line in \cref{fig:simon_msd}), 
suggesting diffusion. As the noise level decreases, the MSD becomes ballistic over short times i.e. there 
is a superdiffusive regime in a small time range where $\text{MSD}(\tau)\propto\tau^2$. The time range of which the MSD is ballistic gets longer 
as the noise level decreases, and at $\eta=0.03\pi$ the superdiffusive regime is longest(blue line in \cref{fig:simon_msd}), where there are fully-fledged channels 
through which the active particles can propagate unhindered, clearly shown by the blue shaded areas, see \cref{fig:simon_noise_003pi}. 

%Although this is only one of the many areas of interest 
%when it comes to active brownian particles. Many other previous 
%studies have shown some interesting behaviors also, in systems 
%with both passive and active Brownian particles. %maybe cite
%Yet the results from \citeauthor{nilsson2017metastable}, 
%regarding the behavior of forming and reusing channels by active particles, 
%seem to be unexplored grounds. The focus of this thesis will be on 
%that particular behavior, whether it is universal and 
%if the results mentioned here can be reproduced in macroscale.

This thesis will be largely based on the results from \citeauthor{nilsson2017metastable} and 
the focus will be especially on the dynamics of mixed systems with passive and active particles. 
We will try to answer whether or not the results above can be reproduced in the macroscale, 
and also further explore the collective behaviors of the active particles, more specifically 
the behavior of forming and reusing channels by the active particles. 

\newpage
\section{Motivation and aim of the thesis}

This thesis aims at showing the universality in the collective motions of active particle systems, 
by studying the behaviors of a system of autonomous agents in a complex environment. 
It will in particular consider how systems of simple agents can lead to very complex behaviors. 
More specifically, we want to answer the question whether the result from \cite{nilsson2017metastable} is 
universal i.e valid in different scales, and whether or not it is possible to design a system in 
macroscale to reproduce these results. There are two behaviors in focus that will be investigated 
in this thesis, first; the phase transition of the active particles from a diffusive state to a ballistic state and 
second; the channel formation and reusing of channels by the active particles. While this is the main subject of 
this project, there are efforts being made in parallel within the Soft Matter lab on closely related projects such as 
experiments with bacterias in a complex environment of obstacles made of glass beads, by Saga Helgadottir, and 
experiments with Janus particles as active particles with $\num{2.6}-$lutidine colloids for passive particles, by 
Giorgio Volpe's and collegues. Being able to compare the results from the different projects with this one 
will lead to a clearer picture of how systems with active and passive particles behave in different 
scales and perhaps answer the question whether the behaviors are universal or not. In fact, similar model 
of active particles have been used in different types of systems of different scales, but it is still a challenge 
for theoretical physics to find minimal statistical models that can capture these features that are inherent 
for active particles systems\cite{toner2005hydrodynamics, li2008minimal, bertin2009hydrodynamic}

The emphasis of studying the autonomous agents in a complex environment is based on the 
fact that self-propelled particles often move in patterned environments, for example E. coli 
inside the intestinal tract\cite{berg2008coli}, or chemotactic bacteria moving through porous 
polluted soils\cite{ford2007role}. Thus conducting experiments using complex environment 
will give a more realistic picture of how real world systems behave.
 
As pointed out before, the focus is to design and study a system consisting of multiple elements/agents, where 
each agent is only able to perform simple task as moving forward or turn when hitting an obstacle. The 
complexity lies in the interaction of these simple agents and their surroundings over time. It will be 
interesting to study how the system evolves over time and also what factor contributes to a certain behavior. 
