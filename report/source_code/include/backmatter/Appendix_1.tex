% CREATED BY DAVID FRISK, 2016
\chapter{Appendix 1}

\section{Theory} \label{theory}

In the following sections, the theory for different equations is explained

\subsection{Mean square displacement (MSD)} \label{theory_msd}

MSD is the most common measure of the spatial extent of random motion, and is measured 
over time to determine if a particle is spreading solely due to diffusion, or if there 
are contributions of other forces\cite{tarantino2014tnf}. The MSD in 1-dimension is defined as

\begin{align}
    MSD(\tau)   &=  \langle (x(t+\tau)-x(t))^2\rangle \\
                &=  \dfrac{1}{N-\tau}\sum_{t=1}^{N}\left( x(\tau+t)-x(t)\right)^2
\end{align}

and in 2-dimension

\begin{align}
    MSD(\tau)   &=  \langle (x(t+\tau)-x(t))^2 + (y(t+\tau)-y(t))^2\rangle \label{eqn:2dim_msd}\\
                &=  \dfrac{1}{N-\tau}\sum_{t=1}^{N}\left( x(\tau+t)-x(t)\right)^2 + \left( y(\tau+t)-y(t)\right)^2
\end{align}

\subsection{Velocity} \label{theory_velocity}

Having the coordinates of the particles in each frame and the frame rate of the camera, 
the velocity can be calculated as follows

\begin{align}
    \begin{array}{lr}
        v_t^i = \dfrac{\sqrt{\left(x_{t+1}^i-x_t^i\right)^2+\left(y_{t+1}^i-y_t^i\right)^2}}{r^{-1}}, 
        & \text{for } t=0,\ldots,F-2\\
        & \text{for } i=0,\ldots,N-1
    \end{array}
    \label{eq:velocity_1par1}
\end{align}

where $F$ is the total number of frames that was tracked for particle $i$, $N$ is the total number of 
particles for the experiment at hand and $r$ is the frame rate. 
The expression above can be simplified further to

\begin{align}
    \begin{array}{lr}
        v_t^i = r\sqrt{\left(x_{t+1}^i-x_t^i\right)^2+\left(y_{t+1}^i-y_t^i\right)^2}, 
        & \text{for } t=0,\ldots,F-2\\
        & \text{for } i=0,\ldots,N-1
    \end{array}
    \label{eq:velocity_1par2}
\end{align}

For particle $i$ we can obtain $F^i-1$ velocity points. To collect the velocity points 
for an experiment for all the frames and all the active particles in one long vector, the
vector will have the dimension $\bm{v}\in\mathbb{R}^{(F-N)\times1}$, and $F$ is the total 
number of frames that was tracked for the specific experiment i.e. $F=\sum_{i=0}^{N-1}F^i$. 
Note that an experiment corresponds to a specific number of active particles $N$, 
specific number of passive particles $C$ and the weight of the passive particles $W$.
Now  the average velocity of the active particles for an experiment can be calculated as

\begin{align}
        \langle\bm{v}\rangle    &=   \sum_{i=0}^{N-1}\sum_{t=0}^{F-2}v_t^i \\
                                &=   r\sum_{i=0}^{N-1}\sum_{t=0}^{F-2}\sqrt{\left(x_{t+1}^i-x_t^i\right)^2+\left(y_{t+1}^i-y_t^i\right)^2}, 
    \label{eq:velocity_average}
\end{align}
